\chapter{Future work}
\label{chapter:future-work}

The results presented in this paper lead to empirical observation when talking about building networks solving different problems. The paper's author proposes new system enhancements based on these experiments.

First of all, the algorithm need to be tested on more complex games where the state of universe is not fully observed(Donkey Kong\footnote{\url{http://www.atariage.com/software_page.php?SoftwareID=990&LabelID=6}}) by our agent or dynamic environments (Asteroids\footnote{\url{https://atariage.com/manual_html_page.php?SoftwareID=828}}) where the scenarios keep changing. In the last case, the reasearch should be done on replay memory technique.

Secondly, we have to test the algorithm on real-life sceanrios. NAO is a humanoid robot\footnote{\url{http://doc.aldebaran.com/1-14/family/robots/motors_robot_v33.html}} built by Aldebaran Robotics\footnote{\url{https://www.aldebaran.com/en}}. In this case we can choose Tic-Tac-Toe game where Nao will play with a human. Every time the game is finished the human will release sounds corresponding on bad rewards and good rewards. Another good part with using Nao is the framework written in Python. This could lead in using another deep learning framework, Theano. Some certain aspect may have been taken into consideration. The network should be trained on different perspectives of the camera, noise reduction should be applied on the frames. Another important aspect is represented by illumination. We could have bad illumination or good illumination. This represents a real life scenarion where the agent will face many difficults and of course the engineer behind the project, but final result may be satisfactory and may lead to a new research idea.

Last but not least, the ideas presented in this paper could lead to development of another project. We could have self-driving cars based on using deep learning in pedestrian detection. Imagine what it would be like to have cars that can see human passing by and avoid accidents.
