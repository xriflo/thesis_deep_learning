\chapter{Conclusions}
\label{chapter:conclusions}

In this thesis we have presented the architecture of a convolutional neural network and optimization techniques for a better learning. Deep learning is considered to be closer to artificial intelligence, it gives the power to networks to learn features, to synthesize the information raw information as pixel values. We have seen that building a network that could learn very well implies a lot of things to be considered like choosing loss function or how many layers a network should have or even how to preprocess data. We have seen that simple model lead to unlearning samples and complex model lead to overfitting. The small number of neurons on the last fully-connected layer and removing the last convolution layer lead to better results. Also, we have seen that RGB color space perform better than YUV on our problem. Another problem that was encountered is the disproportionately values predicted from Q-Learning. We had to apply logarithm function on them to bring in the same range value and then we had to normalize them in order to have a normal distribution. Also, to find the trade-off need multiple experiments in order to achieve the optimum model.

Chapter \ref{chapter:state} \textbf{State of the art} was oriented to gather all necessary information on how the model of architecture should look. Making a parallel between reinforcement learning and convolutional neural networks was the best way to achieve an in-depth overview of the actual algorithms used in deep learning.

Chapter \ref{chapter:related} \textbf{Related work} This chapter revealed the work of people from DeepMind. It described the implementation of the whole system and it gave the starting point on the research.

Chapter \ref{chapter:system-design} \textbf{System design and implementation} presented the implementation of the algorithms discussed in the previous chapter. Every layer of the network was discussed in order to build a solid argument on each choice from preprocessing data to training and testing. The author of the paper chose to combine the architecture implementation with results chapter for the sake of keeping results of each architecture next to its implementation.

We have seen that deep learning can be used efficiently in many problems, from recognizing handwritten digits to cancer classification. This discover could lead to a new era for artificial intelligence. What if we could have an alternative for preventing accidents or what if we could prevent diseases on time?
