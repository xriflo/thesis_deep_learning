% vim: set tw=78 tabstop=4 shiftwidth=4 aw ai:
\documentclass{beamer}

\usepackage[utf8x]{inputenc}		% diacritice
\usepackage[english]{babel}
\usepackage{color}			% highlight
\usepackage{alltt}			% highlight
\usepackage{caption}
\usepackage{frame}


% highlight; comment this out in case you don't input code source files
%\usepackage{code/highlight}		% highlight
\usepackage{hyperref}			% folosiți \url{http://...}
					% sau \href{http://...}{Nume Link}
\usepackage{verbatim}

\mode<presentation>
{ \usetheme{Berlin} }

% Încărcăm simbolurilor Unicode românești în titlu și primele pagini
\PreloadUnicodePage{200}

% Arătăm numărul frame-ului
\newcommand{\frameofframes}{/}
\newcommand{\setframeofframes}[1]{\renewcommand{\frameofframes}{#1}}

\setframeofframes{of}
\makeatletter
\setbeamertemplate{footline}
  {%
    \begin{beamercolorbox}[colsep=1.5pt]{upper separation line foot}
    \end{beamercolorbox}
    
    \begin{beamercolorbox}[ht=2.5ex,dp=1.125ex,%
      leftskip=.3cm,rightskip=.3cm plus1fil]{title in head/foot}%
      {\usebeamerfont{title in head/foot}\insertshorttitle}%
      \hfill%
      {\usebeamerfont{frame number}\usebeamercolor[fg]{frame number}\insertframenumber~\frameofframes~\inserttotalframenumber}
    \end{beamercolorbox}%
    \begin{beamercolorbox}[colsep=1.5pt]{lower separation line foot}
    \end{beamercolorbox}
  }
\makeatother

\setbeamertemplate{navigation symbols}{}%remove navigation symbols

\title[Playing games with Deep Learning]{Playing games with Deep Learning}
\subtitle{Bachelor Thesis Session -- September 2015}
\institute{Faculty of Automatic Control and Computers,\\
	University POLITEHNICA of Bucharest}
\author[Florentina-Ștefania Bratiloveanu]{Florentina-Ștefania Bratiloveanu\\
	Supervisor: As. Drd. Ing Tudor Berariu}
\date{September 14, 2015}

\begin{document}

% Slide-urile cu mai multe părți sunt marcate cu textul (cont.)
\setbeamertemplate{frametitle continuation}[from second]

\frame{\titlepage}

\frame{\tableofcontents}

% NB: Secțiunile nu sunt marcate vizual, ci doar apar în cuprins
\section{Motivation}

% Titlul unui frame se specifică fie în acolade, imediat după \begin{frame},
% fie folosind \frametitle
\begin{frame}{Motivation}
	\begin{itemize}		% Just like normal LaTeX
		\item TODO
	\end{itemize}
\end{frame}

\section{State of the art}
\begin{frame}{Once upon a time...}
\end{frame}
\begin{frame}{Model}
\end{frame}
\begin{frame}{Loss functions and optimizations techniques}
\end{frame}
\begin{frame}{Training and testing}
\end{frame}

\section{Architecture, Design, Results}
\begin{frame}{Once upon a time...}
	\begin{itemize}
		\item Basic formula
			\begin{beamerboxesrounded}[lower=block body,shadow=true]{}
				\texttt{\#define MAX(a, b)   ((a) > (b) ? (a) : (b))}
			\end{beamerboxesrounded}
		\item TODO
	\end{itemize}
\end{frame}



\begin{frame}{Q-Learning}
	\begin{figure}[hp]
\centering
\minipage{0.30\textwidth}
  \framebox{\includegraphics[width=\linewidth]{img/50}}
  \caption*{\newline UP = 90,6534\newline DOWN = 86,8787\newline LEFT = 89,1867\newline RIGHT = 94,2824}\label{fig:fa}
\endminipage\hfill
\minipage{0.30\textwidth}
  \framebox{\includegraphics[width=\linewidth]{img/85}}
  \caption*{\newline UP = 97,6530\newline DOWN = 100,0000\newline LEFT = 93,8538\newline RIGHT = 92,5261}\label{fig:fb}
\endminipage\hfill
\minipage{0.30\textwidth}%
  \framebox{\includegraphics[width=\linewidth]{img/155}}
   \caption*{\newline UP = 26,3520\newline DOWN = 23,8452\newline LEFT = 23,8897\newline RIGHT = 22,8827}\label{fig:fc}
 
\endminipage
\end{figure}
\end{frame}


	
\section{Conclusions}

\begin{frame}{Conclusions}
	\begin{itemize}
		\item TODO
		\item TODO
	\end{itemize}
\end{frame}

\section{Future work}

\begin{frame}{Conclusions}

\begin{figure}[h]
  
  \begin{minipage}[b]{0.3\textwidth}
    \includegraphics[width=\textwidth]{img/joke}
  \end{minipage}
  \hfill
  \begin{minipage}[t]{0.55\textwidth}
	
		\includegraphics[width=\textwidth]{img/training}	   
    
  \end{minipage}
\end{figure}
\end{frame}

\section{Questions}

\begin{frame}{Questions}
  \begin{columns}
    \begin{column}[l]{0.5\textwidth}
      \begin{itemize}
        \item keyword1
        \item keyword2
        \item keyword3
        \item keyword4
        \item keyword5
      \end{itemize}
    \end{column}
    \begin{column}[c]{0.5\textwidth}
      \begin{figure}
        \includegraphics[scale=0.3]{img/question-mark}
      \end{figure}
    \end{column}
  \end{columns}
\end{frame}

\end{document}
